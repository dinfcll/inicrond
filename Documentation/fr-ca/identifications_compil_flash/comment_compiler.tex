%$Id: comment_compiler.tex 46 2005-10-20 16:16:33Z sebhtml $

\documentclass[letterpaper]{article}
\usepackage[utf8]{inputenc}

\begin{document}

\begin{center}
Comment compiler différemment avec Flash selon une variable?\\
(Remplissage de blancs ou Associations sur image)\\

Sébastien Boisvert \\
sebhtml@yahoo.ca\\
20050924\\
\end{center}

\pagebreak
\section{Remplissage de blancs}

\begin{enumerate}


\item 
Ouvrir le fichier "includes/identifications/c\_constants.as" et changer la valeur de la variable \_\_COMPILE\_TYPE\_\_ à 1 \\
ex.: var \_\_COMPILE\_TYPE\_\_ = 1;

\item 
Ouvrir le fichier flash (*.fla) et aller dans la première image de l'étiquette "initialisation"

\item 
Commenter la ligne qui contient l'instruction suivante :\\
\#include "../../includes/prototype\_image/prototype\_image\_func.as"

\item 
Enlever le // devant la ligne qui contient l'instruction suivante :\\
\#include "../../includes/remplir\_bl/func.as"




\item 
Changer le nom du fichier swf (*.swf) avec un nom adéquat suivit de -multiple \\
ex. : introduction\_1\_remplir\_bl-v200508-multiple.swf

\item 
Publier l'animation


\end{enumerate}


\section{Associations sur image}

\begin{enumerate}
\item Ouvrir le fichier "includes/identifications/c\_constants.as" et changer la valeur de la variable \_\_COMPILE\_TYPE\_\_ à 2 \\
ex.: var \_\_COMPILE\_TYPE\_\_ = 2;

\item Ouvrir le fichier flash (*.fla) et aller dans la première image de l'étiquette "initialisation"

\item Commenter la ligne qui contient l'instruction suivante :\\
\#include "../../includes/remplir\_bl/func.as"

\item Enlever le // devant la ligne qui contient l'instruction suivante :\\
\#include "../../includes/prototype\_image/prototype\_image\_func.as"

\item 
Changer le nom du fichier swf (*.swf) avec un nom adéquat suivit de -multiple \\
ex. : introduction\_1\_identifications\_images-v200508-multiple.swf

\item 
Publier l'animation

\end{enumerate}

\vspace{1cm}
\hrule
Le suffixe -multiple indique que le swf a été généré avec le fichier flash source (*.fla) qui peut générer à la fois Remplissage de blancs et Associations sur image.

\end{document}

